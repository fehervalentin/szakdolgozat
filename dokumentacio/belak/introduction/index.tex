% Introduction
\section{Bevezetés}

\subsection{A témaválasztás indoklása}
Mindenképpen egy online, többfelhasználós játékot szerettem volna megvalósítani. Később leszűkítettem a kört kártyajátékokra, és végül az ulti és a póker között vaciláltam. A döntésem a pókerre esett, ugyanis az ulti viszonylag bonyolultabb, mint a póker, több szabály, több megszorítás a partykra vonatkozólag, ráadásul ahány ház annyi szokás alapon könnyen nézet eltérések szoktak keletkezni az ultizás során. Pókerezni egyszerűbb - bár ezt sokan vitatják - és jó formán mindenki könnyen megérti a játék lényegét.

\subsection{A megoldandó feladat leírása}
A játék általam elképzelt reprezentációja egy szerver- és egy kliens alkalmazásból
áll, melyeknek jól meghatározott feladata van. Ezek egymástól függetlenül kerülnek
fejlesztése úgy, hogy a kliens csak egy jól megfogalmazott API-t ismer a szervertől,
melyen keresztül kommunikálni tudnak. Ennek az API-nak a feladata, hogy egy felületet
biztosítson arra, hogy a felhasználók a kliens alkalmazás használatával:

\begin{itemize}
  \item be tudjanak jelentkezni a szerverre, vagy arra regisztrálni tudjanak
  \item megtalálják a szerveren elérhető játékosokat, barátokat
  \item megtekinthessék a ranglistát
  \item új játékot tudjanak indítani, vagy egy már megkezdett játékot folytatni, amennyiben az abban játszó összes játékos elérhető
  \item a játékban lehetséges műveleteket végre tudják hajtani, a változásokról értesítést kaphassanak
  \item játék közben egymásnak szöveges üzenetet küldhessenek
  \item egy véget ért játékot le tudjanak menteni
\end{itemize}

Az API pontos megfogalmazása után a szerver- és a kliens alkalmazást egymástól
teljesen külön kezelhetjük.

\subsubsection{A szerver alkalmazás leírása}
A szerver alkalmazás egyrészt egy webes felületet nyújt az adminisztrátorok számára,
melyen keresztül a felhasználókat és a játékokat, eredményeket valamint a játékhoz
tartozó naplókat kezelhetik. \\
Másrészt pedig a szerver alkalmazás a teljes játékot (a játék logikáját és a
lekérdezett játék aktuális állapotát), mint szolgáltatást nyújtja a fent említett
API-n keresztül, melyet a kliensek autentikálás után használhatnak. \\
A szerver párhuzamosan egyszerre több játékot is tud futtatni és egy játék a
játékosok kérésére bármikor felfüggesztésre, majd később folytatásra kerülhet.\\
A szerver rendelkezik egy ranglistával is, melyen a játékosok az alábbi módon
kerülhetnek feljebb:

\begin{itemize}
  \item 10 pontot kap az a játékos, aki megnyer egy játékot
  \item  5 pontot kap az a játékos, aki ugyan tudta a megoldást, de egy másik játékos megelőzte a vádolás folyamatában
\end{itemize}

\subsubsection{A kliens alkalmazás leírása}
A szakdolgozatom kliens alkalmazása egy olyan asztali alkalmazás, amely egy jól
átlátható felületet nyújt a játékhoz. Elsődleges célja, hogy a szerver API-ját
használva lekérdezze a játék állapotát, majd a kapott állapotot megjelenítse,
valamint, hogy a felhasználói interakciót a szerver felé továbbítsa és az
eredményt megjelenítse. \\
Magán a játék felületen kívül a kliens alkalmazáson keresztül a felhasználó
új játékot hozhat létre barátok, vagy online játékosok kiválasztásával, kezelheti
saját profilját és beállításait, valamint megtekintheti a szerver ranglistáját.\\
A kliens alkalmazás ezen felül lehetőséget ad a felhasználónak arra, hogy lementett
játékokat játsszon vissza.

\cleardoublepage
