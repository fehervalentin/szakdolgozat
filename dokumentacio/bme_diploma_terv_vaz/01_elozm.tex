%%%%%%%%%%%%%%%%%%%%%%%%%%%
% Szennycimoldal
%%%%%%%%%%%%%%%%%%%%%%%%%%%
 \begin{center}
 \pagenumbering{Roman}

 \vspace*{12mm}
 \LARGE{Diplomaterv}

 \vspace{8mm}
 \large{\nev}
 \vfill
 \ev
 \end{center}
 \thispagestyle{empty}
 \blankpage

%%%%%%%%%%%%%%%%%%%%%%%%%%%
% Diplomaterv-kiiras (ezt adjak, bele kell k?tni a diplom?ba)
%%%%%%%%%%%%%%%%%%%%%%%%%%%
\begin{center}

 \uppercase{Budapesti M?szaki ?s Gazdas?gtudom?nyi Egyetem}\\
 \uppercase{Villamosm?rn?ki ?s Informatikai Kar}\\
 \uppercase{M?r?stechnika ?s Inform?ci?s Rendszerek Tansz?k}

 \vspace{6mm}
 \large\textbf{DIPLOMATERV FELADAT (ezt adj?k\dots)}

 \vspace{6mm}
 \textbf{\nev} \\ \strut \\
 szigorl? villamosm?rn?k hallgat? r?sz?re\\
 (nappali tagozat villamosm?rn?ki szak) \\

 \vspace{6mm}
 \textbf{\cim}\\
 (a feladat sz?vege a mell?kletben)

\end{center}

 \vspace{6mm}
 A tervfeladatot ?ssze?ll?totta ?s a tervfeladat
tansz?ki konzulense:

\begin{center}
\konzulens\\ \konzbeoszt
\end{center}

 \vspace{6mm}
 \begin{tabular}{p{80mm}l}
A z?r?vizsga t?rgyai:   & Els? t?rgy \\
                        & M?sodik t?rgy \\
                        & Harmadik t?rgy
 \end{tabular}

 \vspace{6mm}
 \begin{tabular}{p{80mm}l}
 A tervfeladat kiad?s?nak napja:         &  \\
 A tervfeladat bead?s?nak hat?rideje:    &
 \end{tabular}

\vfill

\begin{center}
\begin{tabular}{cc}
 \makebox[7cm]{\emph{dr.\ G?rg?nyi Andr?s}}    & \makebox[7cm]{\emph{dr.\ P?celi G?bor}} \\
 \makebox[7cm]{adjunktus, diplomaterv felel?s} & \makebox[7cm]{egyetemi tan?r, tansz?kvezet?}
\end{tabular}
\end{center}

 \vspace{6mm}
 \begin{tabular}{p{80mm}l}
 A tervet bevette:           & \\
 A terv bead?s?nak d?tuma:   & \\
 A terv b?r?l?ja:            &
 \end{tabular}


 \thispagestyle{empty}
 \blankpage


%%%%%%%%%%%%%%%%%%%%%%%%%%%
% Diplomaterv-kiiras melleklete (ezt is adjak, bele kell k?tni a diplom?ba)
%%%%%%%%%%%%%%%%%%%%%%%%%%%
\def\abstractname{Mell?klet}
\begin{abstract}

\begin{center}
\textbf{\cim}
\end{center}

Itt k?vetkezik a r?szletes feladatki?r?s

Szint?n el?re k?szen van, itt csak a helyet hagyjuk ki.


\begin{flushright}
 \vspace*{1cm}
 \makebox[7cm]{\konzulens}\\
 \makebox[7cm]{\konzbeoszt}
\end{flushright}
\end{abstract}

 \thispagestyle{empty}
 \blankpage

%%%%%%%%%%%%%%%%%%%%%%%%%%%
% Nyilatkozat
%%%%%%%%%%%%%%%%%%%%%%%%%%%
\def\abstractname{Nyilatkozat}
\begin{abstract}

Alul?rott  \emph{\nev},  a  Budapesti M?szaki ?s Gazdas?gtudom?nyi
Egyetem hallgat?ja kijelentem,  hogy  ezt  a diplomatervet meg nem
engedett seg?ts?g n?lk?l, saj?t  magam k?sz?tettem,  ?s a
diplomatervben csak a megadott forr?sokat haszn?ltam  fel. Minden
olyan  r?szt, amelyet  sz?  szerint,  vagy azonos ?rtelemben, de
??tfogalmazva  m?s forr?sb?l ??tvettem, egy?rtelm?en, a forr?s
megad?s?val megjel?ltem.

\begin{flushright}
 \vspace*{1cm}
 \makebox[7cm]{\rule{6cm}{.4pt}}\\
 \makebox[7cm]{\emph{\nev}}\\
 \makebox[7cm]{hallgat?}
\end{flushright}
\end{abstract}

%%%%%%%%%%%%%%%%%%%%%%%%%%%
% Tartalomjegyzek
%%%%%%%%%%%%%%%%%%%%%%%%%%%
\tableofcontents

%%%%%%%%%%%%%%%%%%%%%%%%%%%
% Kivonat
%%%%%%%%%%%%%%%%%%%%%%%%%%%
\def\abstractname{Kivonat}
\begin{abstract}
\addcontentsline{toc}{chapter}{Kivonat}

A diplomaterv st?lusf?jlokkal foglalkozik.

Bemutatja, elemzi, kieg?sz?ti ?s ?sszefoglalja a st?lusok
haszn?lat?nak m?dj?t.

Egy- ill.\ k?toldalas opci?val is j?l m?k?dik.
\end{abstract}


%%%%%%%%%%%%%%%%%%%%%%%%%%%
% Abstract
%%%%%%%%%%%%%%%%%%%%%%%%%%%
\selectlanguage{english}
\begin{abstract}
\addcontentsline{toc}{chapter}{Abstract}

This master thesis discusses \LaTeXe\ style files for BUTE
undergraduate students.
\end{abstract}

\selectlanguage{magyar}
