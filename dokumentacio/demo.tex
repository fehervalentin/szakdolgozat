%demonstrations of Prosper capabilities
%written by Janos Markus, markusjancsi@users.sourceforge.net

%Copyright (c) Janos Markus (markusjancsi@users.sourceforge.net), BUTE, 2002. 
%All rights reserved.

%% WARNING: this demonstration file is special in the sense that in includes
%% some ps-figures at some special coordinates (namely the examples).
%% This file won't look good with any other style-file or without
%% letterpaper option! Sorry.

%% Due to the large size of the example ps files, I did not include them with
%% the source. However, if you really want to generate the same output,
%% you can generate the Example*.ps figures first from their *.tex 
%% counterpair, then compile this code.

\documentclass[pdf,letterpaper,colorBG,slideColor,gradboy]{prosper}

\usepackage{amsmath}
\usepackage{graphicx}
\usepackage{epsfig}
\usepackage{pstricks}

\newpagestyle{examplepagestyle}{}{}
\renewcommand\textregistered{\textsuperscript{\textcircled{\usefont{OT1}{cmr}{m}{sc}\selectfont r}}}
\renewcommand{\to}{\ensuremath\rightarrow}
\newcommand{\textcode}[1]{{\usefont{OT1}{cmtt}{m}{n}\selectfont #1}}


\Logo{\includegraphics[width=1.4\realin, keepaspectratio]{logo2}}

\title{Demonstration of the\\ \textcode{prosper} class}
\subtitle{v 1.9a}
\author{J\'anos M\'arkus}
\email{markusjancsi@users.sourceforge.net}
\institution{%
Oregon State University\\
Department of Electrical and Computer Engineering}

\slideCaption{The \textcode{prosper} class}

\begin{document}

%============================ SLIDE ============================
\addtocounter{slide}{-1}
\begin{slide}{Technical Comments}

\begin{itemize}
\item To switch to full-screen, use the option \texttt{View \to\ Full Screen}, 
to leave, hit the \texttt{$<$Esc$>$} key;
\item If some superscript is blocked out by a gray box in the following 
equation: \tiny $\left(x+y\right)^{\left(a_i-b_{i+1}+c_{i-1}\right)^{\left(2+e^{x+a}\right)}}$,
\small then
% in Acrobat\textregistered\ Reader\texttrademark\ 
uncheck the \texttt{Edit \to\ Preferences \to\ Display \to\ Use Greek Text} 
option;
\item \texttt{PgUp}, \texttt{PgDn} or the \texttt{arrow} keys advance pages,
however, in the \texttt{Preferences \to\ Full Screen} section you can 
set the mouse to advance, too. (You can also hide it, overwrite default 
transition and many more.)
\end{itemize}
\end{slide}


%============================ SLIDE ============================
\maketitle

%============================ SLIDE ============================
%\begin{slide}{Corner testing}%
%
%\pscircle*[linecolor=green,fillcolor=green](-1,2.1){.08}
%\pscircle*[linecolor=green,fillcolor=green](10,2.1){.08}
%
%\rule{1ex}{1ex}
%\end{slide}

%============================ SLIDE ============================
\overlays{6}{%
\begin{slide}{Introduction}%
The \textcode{prosper} class (written by Fr\'ed\'eric Goualard)
\begin{itemstep}
\item permits producing high quality slides;
\item offers all the advantages (and disadvantages) of \LaTeX;
\item freely available;
\item platform independent;
\item easily extendable;
\item make the task of recycling materials from previously written 
article(s) easy.
\end{itemstep}
\end{slide}}

%============================ SLIDE ============================
\begin{slide}{Prerequisites}%
\textcode{Prosper} relies on some recent \LaTeX\ packages and other softwares. 
The following version-numbers are mandatory:

\begin{itemize}
  \item a recent distribution of \LaTeX\ with the \textcode{pstricks} 
        and \textcode{seminar} packages with \textcode{seminar.bg2} (May 22, 1998);
  \item a recent version of \textcode{hyperref} (version $\ge$ 6.70u);
  \item \textcode{dvips} (version $\ge$ 5.85);
  \item \textcode{GhostScript} (version $\ge$ 6.5) for handling embedded fonts.
\end{itemize}
\end{slide}

%============================ SLIDE ============================
\overlays{3}{
\begin{slide}{Styles}%

\textcode{Prosper} offers some predefined and contributed styles to prepare a 
presentation. These are as follows:

\begin{itemize}
\fromSlide{2}{\item Predefined Styles

\textcode{default}, \textcode{alienglow}, \textcode{autumn}, \textcode{azure}, 
\textcode{contemporain}, \textcode{darkblue}, \textcode{frames}, 
\textcode{lignesbleues}, \textcode{nuancegris}, \textcode{troispoints};}

\fromSlide{3}{\item Contributed Styles

\textcode{alcatel} (based on \textcode{troispoints}), \textcode{blends}, 
\textcode{gradboy}, \textcode{pascal}, \textcode{rico}, \textcode{gyom}.}

\end{itemize}
\end{slide}}

%============================ SLIDE ============================
\overlays{7}{
\begin{slide}{Transitions}%
The following transitions (switching effects between two slides) 
are defined:
\begin{itemstep}
\item R or Replace;
\item Split;
\item Blinds;
\item Box;
\item Wipe;
\item Dissolve;
\item Glitter.
\end{itemstep}
\end{slide}}

\iffalse
%============================ SLIDE ============================
\begin{slide}{Predefined styles}%

The next 10 slides show the different styles developed by the author of 
the class-file (Fr\'ed\'eric Goualard). 

\medskip

Each slide's title contains the stylename and the transition's name 
which is used to switch to that particular slide.

\end{slide}

\slidepagestyle{examplepagestyle}
\Logo{}
%============================ SLIDE ============================
\begin{slide}{Default style w/ Replace}
\rput[lt](-1,2.1){%
\psframebox[fillstyle=solid,fillcolor=white,framesep=0,
linewidth=0pt,linestyle=none]{\includegraphics[angle=-90, 
width=\RealPaperWidth]{examples/Example.ps}}}
\end{slide}

%============================ SLIDE ============================
\begin{slide}{Alien glow style w/ Split}
\rput[lt](-1,2.1){%
\psframebox[fillstyle=solid,fillcolor=white,framesep=0,
linewidth=0pt,linestyle=none]{\includegraphics[angle=-90, 
width=\RealPaperWidth]{examples/ExampleAlienGlow.ps}}}
\end{slide}

%============================ SLIDE ============================
\begin{slide}{Autumn style w/ Blinds}
\rput[lt](-1,2.1){%
\psframebox[fillstyle=solid,fillcolor=white,framesep=0,
linewidth=0pt,linestyle=none]{\includegraphics[angle=-90, 
width=\RealPaperWidth]{examples/ExampleAutumn.ps}}}
\end{slide}

%============================ SLIDE ============================
\begin{slide}{Azure style w/ Box}
\rput[lt](-1,2.1){%
\psframebox[fillstyle=solid,fillcolor=white,framesep=0,
linewidth=0pt,linestyle=none]{\includegraphics[angle=-90, 
width=\RealPaperWidth]{examples/ExampleAzure.ps}}}
\end{slide}

%============================ SLIDE ============================
\begin{slide}{Contemporain style w/ Wipe}
\rput[lt](-1,2.1){%
\psframebox[fillstyle=solid,fillcolor=white,framesep=0,
linewidth=0pt,linestyle=none]{\includegraphics[angle=-90, 
width=\RealPaperWidth]{examples/ExampleContemporain.ps}}}
\end{slide}

%============================ SLIDE ============================
\begin{slide}{Dark blue style w/ Dissolve}
\rput[lt](-1,2.1){%
\psframebox[fillstyle=solid,fillcolor=white,framesep=0,
linewidth=0pt,linestyle=none]{\includegraphics[angle=-90, 
width=\RealPaperWidth]{examples/ExampleDarkBlue.ps}}}
\end{slide}

%============================ SLIDE ============================
\begin{slide}{Frames style w/ Glitter}
\rput[lt](-1,2.1){%
\psframebox[fillstyle=solid,fillcolor=white,framesep=0,
linewidth=0pt,linestyle=none]{\includegraphics[angle=-90, 
width=\RealPaperWidth]{examples/ExampleFrames.ps}}}
\end{slide}

%============================ SLIDE ============================
\begin{slide}{Lignes bleues style w/ Replace}
\rput[lt](-1,2.1){%
\psframebox[fillstyle=solid,fillcolor=white,framesep=0,
linewidth=0pt,linestyle=none]{\includegraphics[angle=-90, 
width=\RealPaperWidth]{examples/ExampleLignesBleues.ps}}}
\end{slide}

%============================ SLIDE ============================
\begin{slide}{Nuances de gris style w/ Blinds}
\rput[lt](-1,2.1){%
\psframebox[fillstyle=solid,fillcolor=white,framesep=0,
linewidth=0pt,linestyle=none]{\includegraphics[angle=-90, 
width=\RealPaperWidth]{examples/ExampleNuanceGris.ps}}}
\end{slide}

%============================ SLIDE ============================
\begin{slide}{Trois points style w/ Box}
\rput[lt](-1,2.1){%
\psframebox[fillstyle=solid,fillcolor=white,framesep=0,
linewidth=0pt,linestyle=none]{\includegraphics[angle=-90, 
width=\RealPaperWidth]{examples/ExampleTroisPoints.ps}}}
\end{slide}

\slidepagestyle{GenericPageStyle}
\Logo{\includegraphics[width=1.4\realin, keepaspectratio]{logo2}}
%============================ SLIDE ============================
\begin{slide}{Contributed styles}%

The next 6 slides show the styles developed by different contributors.

\medskip

Comment: \textcode{Alcatel} slide is based on \textcode{troispoints}.  
Using the \textcode{slideColor} option they look exactly the same.

\end{slide}

\slidepagestyle{examplepagestyle}
\Logo{}
%============================ SLIDE ============================
\begin{slide}{Alcatel style w/ Wipe}
\rput[lt](-1,2.1){%
\psframebox[fillstyle=solid,fillcolor=white,framesep=0,
linewidth=0pt,linestyle=none]{\includegraphics[angle=-90, 
width=\RealPaperWidth]{examples/contrib/ExampleAlcatel.ps}}}
\end{slide}

%============================ SLIDE ============================
\begin{slide}{Blends style w/ Dissolve}
\rput[lt](-1,2.1){%
\psframebox[fillstyle=solid,fillcolor=white,framesep=0,
linewidth=0pt,linestyle=none]{\includegraphics[angle=-90, 
width=\RealPaperWidth]{examples/contrib/ExampleBlends.ps}}}
\end{slide}

%============================ SLIDE ============================
\begin{slide}{Gradient Blue-Orange-Yellow style w/ Glitter}
\rput[lt](-1,2.1){%
\psframebox[fillstyle=solid,fillcolor=white,framesep=0,
linewidth=0pt,linestyle=none]{\includegraphics[angle=-90, 
width=\RealPaperWidth]{examples/contrib/ExampleGradBoy.ps}}}
\end{slide}

%============================ SLIDE ============================
\begin{slide}{Gym personal style w/ Replace}
\rput[lt](-1,2.1){%
\psframebox[fillstyle=solid,fillcolor=white,framesep=0,
linewidth=0pt,linestyle=none]{\includegraphics[angle=-90, 
width=\RealPaperWidth]{examples/contrib/ExampleGyom.ps}}}
\end{slide}

%============================ SLIDE ============================
\begin{slide}{Pascal style w/ Split}
\rput[lt](-1,2.1){%
\psframebox[fillstyle=solid,fillcolor=white,framesep=0,
linewidth=0pt,linestyle=none]{\includegraphics[angle=-90, 
width=\RealPaperWidth]{examples/contrib/ExamplePascal.ps}}}
\end{slide}

%============================ SLIDE ============================
\begin{slide}{Rico style w/ Blinds}
\rput[lt](-1,2.1){%
\psframebox[fillstyle=solid,fillcolor=white,framesep=0,
linewidth=0pt,linestyle=none]{\includegraphics[angle=-90, 
width=\RealPaperWidth]{examples/contrib/ExampleRico.ps}}}
\end{slide}


\fi
\slidepagestyle{GenericPageStyle}
\Logo{\includegraphics[width=1.4\realin, keepaspectratio]{logo2}}
%============================ SLIDE ============================
\begin{slide}{Using the class}

To start using the class, have a look into the following resources:
\begin{itemize}
\item The \href{http://prosper.sourceforge.net/}{{\green homepage}} of the project;
\item The documentation of the class (\textcode{prosper-doc.pdf});
\item Some addition to the documentation -- e.g.\ introducing
      \textcode{letterpaper} option (\textcode{prosper-doc-add.pdf});
\item Example files in the \textcode{doc} directory;
\item The source of this file (\textcode{demo.tex}).
\end{itemize}
\end{slide}



%============================ SLIDE ============================
\begin{slide}{Testing your workflow}

This is a must.

There is a very up-to-date testfile on CTAN, which checks all critical
part of your workflow to determine whether your current configuration
is capable to produce high-quality ps/pdf files or not. 

The following URL leads to the directory which has the testfile 
(\textcode{testflow.tex}), ideal output files and a documentation 
how to detect and correct any errors:

\begin{center}
\href{http://www.ctan.org/tex-archive/macros/latex/contrib/supported/IEEEtran/testflow/}%
{{\green URL: testflow directory on CTAN}}
\end{center}

Read the documentation (\textcode{testflow\_doc.txt}) how to use this test.

\end{slide}

%============================ SLIDE ============================
\begin{slide}{Recycling xfig-figures (1)}

By default, the combined \LaTeX\ and PostScript output of \textcode{xfig}
gives a transparent background. However, in your paper the figure
is probably black. If you include it, it will look like this:

\centering
\resizebox{0.7\slideWidth}{!}{\input{xfig_normal.eps_t}}

\textcode{xfig\_normal.eps\_t}

\end{slide}

%============================ SLIDE ============================
\begin{slide}{Recycling xfig-figures (2)}

Quick-and-dirty trick:

\begin{itemize}
\item Change the fill colors of dots, etc.\ to white;
\item Glue every object into one compound;
\item Update object, set only PenColor option on;
\item Change PenColor to White;
\item Apply it to the compound you made (you won't see too much from now on);
\item Save it to a different file and export it, then include it here.
\end{itemize}

\end{slide}

%============================ SLIDE ============================
\begin{slide}{Recycling xfig-figures (3)}

The text in your figure would inherit the color of the slide text, but, 
you can make it white with the \textcode{\textbackslash white} command.

\begin{center}
{\white
\resizebox{0.5\slideWidth}{!}{\input{xfig_transp.eps_t}}}

\textcode{xfig\_transp.eps\_t}
\end{center}

Well, actually you need both versions, one for color and one 
for white background\dots

\end{slide}

%============================ SLIDE ============================
\begin{slide}{Recycling MATLAB-figures (1)}

\begin{itemize}

\item Use \textcode{plottransp.m} function shipped with this document;
\item Use it instead of \textcode{plot};
\item One can modify it to be used instead of \textcode{semilogx}, etc.;
\item Latest version is 
      \href{http://mit.bme.hu/~markus/latex/prosper/}{{\green here}}.
\end{itemize}
\end{slide}


%============================ SLIDE ============================
\begin{slide}{Recycling MATLAB-figures (2)}

\tiny
\begin{verbatim}
plot(0:.01:1,[(0:.01:1);(0:.01:1).^2;(0:.01:1).^3]);
title('Normal MATLAB figure');
print -deps2c matlab_normal.eps 
\end{verbatim}
\bigskip
\begin{center}
\includegraphics[width=0.5\slideWidth, keepaspectratio]{matlab_normal}

\medskip
\textcode{matlab\_normal.eps}
\end{center}
\end{slide}

%============================ SLIDE ============================
\begin{slide}{Recycling MATLAB-figures (3)}

\tiny
\begin{verbatim}
plottransp(0:.01:1,[(0:.01:1);(0:.01:1).^2;(0:.01:1).^3]);
title('Inverted, transparent MATLAB figure');
print -deps2c matlab_transp.eps 
\end{verbatim}
\bigskip
\begin{center}
\includegraphics[width=0.5\slideWidth, keepaspectratio]{matlab_transp}

\medskip
\textcode{matlab\_transp.eps}
\end{center}
\end{slide}

%============================ SLIDE ============================
\begin{slide}{Recycling Simulink-diagrams (1)}

\begin{itemize}
\item You can print into eps-file by using the 
\verb+print -s<modelname> -deps2c <filename.eps>+ command.
\item Unfortunately, Simulink does not know ``transparent colors'';
\item Simulink 3.0.1 (R11.1) does not even print the background (screen)
color;
\item Simulink 4.1 (R12.1) does. 
\end{itemize}
\end{slide}


%============================ SLIDE ============================
\begin{slide}{Recycling Simulink-diagrams (2)}

To make the Simulink \textcode{eps} file transparent, one have to find the
command which fills up the background with white or any other color.
It is relative easy to find (at least in these versions).

\begin{itemize}

\item Open the printed eps-file in a text editor;
\item Search for the following line:\\
      \textcode{12 dict begin \%Colortable dictionary}\\
      (or part of it, like \textcode{Colortable});
\end{itemize}
\end{slide}

%============================ SLIDE ============================
\begin{slide}{Recycling Simulink-diagrams (3)}

\begin{itemize}
\item under 8 lines of color definitions, you can see the following 
two or three lines:

{\footnotesize

  \begin{tabular}{ll}
  \textcode{\red c0}     & this selects black color; \\
  \red something         & \parbox[t]{.6\slideWidth}{%
                           if your background color is not black,
                           this line sets up the background color;}\\
  \textcode{\red -1 -1 $n_1$ $n_2$ rf}
                         & \parbox[t]{.5\slideWidth}{%
                           where $n_1$ and $n_2$ are 
                           (usually three-digit) numbers.}
  \end{tabular}
}
\item The last line (\textcode{rf}=\textcode{rectfill}) creates the box to 
be filled. If you uncomment this line using the \% character, it results 
a nice transparent figure.

\end{itemize}
\end{slide}

%============================ SLIDE ============================
\begin{slide}{Recycling Simulink-diagrams (4)}

\begin{center}

\includegraphics[angle=-90, width=0.5\slideWidth, keepaspectratio]{simtest_orange}

\medskip
\textcode{simtest\_orange.eps}

\bigskip
\includegraphics[angle=-90, width=0.5\slideWidth, keepaspectratio]{simtest_transp}

\medskip
\textcode{simtest\_transp.eps}

\end{center}
\end{slide}


%============================ SLIDE ============================
\begin{slide}{Writing new style}

To write a new style, the following resources are available:
\begin{itemize}
\item The \href{http://prosper.sourceforge.net/}{{\green homepage}} of the classfile;
\item The documentation of the class (\textcode{prosper-doc.pdf});
\item Some addition to the documentation (\textcode{prosper-doc-add.pdf});
\item various style-files (\textcode{PPR*.sty}).
\end{itemize}
\end{slide}

%============================ SLIDE ============================
\begin{slide}{Acknowledgement}

Many thanks to those, who has contributed to the \textcode{prosper} class:

\begin{itemize}

\item Timothy Van Zandt (author of \textcode{pstricks} and \textcode{seminar}
classes);

\item Denis Girou (who is currently maintaining the previous packages);

\item Fr\'ed\'eric Goualard and the Prosper development team (author 
and developer(s) of the \textcode{prosper} class).

\end{itemize}
\end{slide}

\end{document}
