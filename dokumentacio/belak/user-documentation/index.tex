\part{Felhasználói dokumentáció}
A programcsomagot JAVA programozási nyelven írtam meg, így a program számára biztosítani kell JAVA futtatókörnyezetet, továbbá a program működéséhez adatbázis szerverre is szükség lesz. A telepítés magába foglalja az adatbázis séma elkészítését és az adatbázis demó adatokkal való feltöltését is, így a telepítést követően extra dolgokra nem lesz szükség a szoftver használatához. A telepítés közben a számítógépet újraindítani nem szükséges.

\section{Telepítés}
\subsection{Java SE Runtime Environment 8}
A programcsomag futtatásához legalább Windows XP operációs rendszer szükséges, amelyen Java SE Runtime Environment 8 futtató környezet \cite{jresite} (a továbbiakban: JRE) fut. A JRE feltelepítését követően manuálisan ellenőrizzük, hogy a rendszer felvette-e környezeti változóként az installációs könyvtárat. Navigáljunk az operációs rendszerben a környezeti változók módosítása panelhez, majd ellenőrizzük le, hogy a PATH nevű környezeti változóhoz hozzá lett-e adva az installációs könyvtár, mint például: \path{C:\Program Files\Java\jre1.8.0_66\bin}. A telepítési könyvtár operációs rendszerenként eltérhet.

\begin{figure}[h!]
  \caption{PATH környezeti változó értéke}
  \label{fig:path_env}
  \centering
    \includegraphics{user-documentation/images/path_env.jpg}
\end{figure}

Ha a JRE installációs könyvtár gyökerében található bin könyvtár nincs behivatkozva a PATH nevű környezeti változó értékéhez, akkor az \ref{fig:path_env}. ábrán látható módon egészítsük ki az értéket, majd mentsük el a változtatásokat és indítsuk újra a promptot. \\
Ha sikeresen jártunk el, akkor a

\begin{verbatim}
java -version
\end{verbatim}
utasítás hatására a
\begin{figure}[h!]
  \caption{JRE verzió}
  \label{fig:jre_version}
  \centering
    \includegraphics{user-documentation/images/java_version.jpg}
\end{figure}
 \ref{fig:jre_version}. ábrán látható szöveg jelenik meg a konzolon, akkor sikeres volt a JRE telepítése és beállítása.
 
 \subsection{MySQL Community Server 5.6}
 A programcsomag megköveteli a MySQL Community Server 5.6 adatbázis-kezelő rendszer \cite{mysqlsite} (a továbbiakban: MySQL Server) használatát is. Letöltés után csomagoljuk ki a zip állományt egy tetszőleges könyvtárba, majd a fentiekkel megegyező módon adjuk hozzá a PATH nevű környezeti változó értékéhez a MySQL Server bin könyvtár elérési útvonalát. Ha ezzel végeztünk, akkor nyissük meg a promptot (ha nyitva van, akkor indítsuk újra), majd navigáljunk a MySQL Server bin könyvtárába, ott pedig adjuk ki a
 \begin{verbatim}
mysqld --install
\end{verbatim}
parancsot. A parancs végrehajtása után navigáljunk a szolgáltatások panelhez, amelyet a legkönnyebben a promptban a
 \begin{verbatim}
services.msc
\end{verbatim}
\begin{figure}[h!]
  \caption{MySQL Service}
  \label{fig:mysql_service}
  \centering
    \includegraphics{user-documentation/images/mysql_service.jpg}
\end{figure}
kiadott utasítással lehet elérni. Majd járjunk el a \ref{fig:mysql_service}. ábrának megfelelően. Térjünk vissza a konzolra, ahol adjuk ki a 
 \begin{verbatim}
mysql -u root -p
\end{verbatim}
parancsot, amely jelszót fog kérni. A beviteli sort hagyjuk üresen, nyomjunk entert. Ha sikeresen beléptünk az adatbázis-kezelő rendszerbe, akkor adjuk ki a
 \begin{verbatim}
SELECT VERSION();
\end{verbatim}
utasítást, és ha a \ref{fig:mysql_service}. ábrának megfelelő képernyőképet kapunk, akkor sikeresen feltelepítettük az adatbázis-kezelő rendszert.

\subsection{Az adatbázis használatba vétele}
Ha sikeresen elindítottuk a MySQL Servert, akkor szükségünk lesz egy új adatbázis sémára és demó adatokra, amelyet a \path{X:\poker\release\poker-db.sql} állományban találunk. Ezt a filet kell lefuttatni az adatbázison, a hatása idempotens. A promptban adjuk ki a
 \begin{verbatim}
mysql -u root -p < X:\poker\release\poker-db.sql
\end{verbatim}
utasítást, amely jelszót fog kérni. A beviteli sort ugyancsak hagyjuk üresen. Ha sikeresen lefutott a parancs, akkor az adatbázis séma elkészült és a program demó adatokkal is feltöltötte.

\section{Fogalomtár}
Az alábbi parancsokat a felhasználók tudják kiadni. A szerver biztosítja, hogy csak a megfelelő parancsok legyenek hívhatóak a póker partik során.
\subsection{Parancsok}
\pokerparagraph{CALL}
A felhasználó ezt a parancsot adja ki, ha a tartozását ki szeretné egyenlíteni az asztal felé.
\pokerparagraph{CHECK}
Játékos által kiadható parancs, mely akkor adható ki, amikor a felhasználó nem tartozik az asztal felé, és emelni sem szeretne az adott játékos.
\pokerparagraph{BLIND}
Olyan parancs, amelyet a kliens oldali program automatikusan hív meg. Ha a szerver BLIND típusú utasítást küld, akkor a kliens automatikusan beteszi a kis vagy nagy vakot az asztalra attól függően, hogy a szerver mire kötelezte az adott felhasználót. A kis vak, vagy angol nevén small blind az asztalnál érvényben lévő alaptét felét, míg a nagy vak, vagy angolul big blind az alaptét egészét jelenti.
\pokerparagraph{RAISE}
Olyan speciális parancs, amely hatására a felhasználó tartozása kiegyenlítésre kerül az asztal felé és még emel a téten az asztal alaptétjének megfelelő értékkel.
\pokerparagraph{QUIT}
Asztal elhagyására szolgáló parancs. A kilépését követően a program visszairányítja a felhasználót a táblalistázó oldalra.
\pokerparagraph{CHANGE}
A parancs csak a classic játékstílusban érhető el. A parancs meghívásával a felhasználó utasítja a programot, hogy a cserére megjelölt kártyalapokat a szerver új kártyalapokkal helyettesítse.
\pokerparagraph{LOG}
A felhasználó játék közben vissza tudja nézni a már megtörtént főbb eseményeket. Például az előző játékos milyen parancsot hajtott végre.
\subsection{Egyéb...?}
\pokerparagraph{Dealer}
Magyarul osztó. A játék során az osztó gomb az óramutató járásával megegyező irányban halad körbe az asztalon, így minden parti esetén más felhasználó tölti be az a osztó szerepét.

\section{A póker játékról}
A póker, mint kártyajáték igen népszerű szerencsejáték. Akár élő tv adásokat is végig lehet követni, ahol hatalmas főnyereményeket osztanak ki a dobogós helyezetteknek. Viszonylag sok fajtája terjedt el szerte a világon, kezdve a klasszikus 5 lapos leosztásokkal egészen az OMAHA-án át a jól ismert Texas Hold'Em játékstílusig. A játékot 52 lapos francia kártyapaklival játszák, amelyben 4 szín és 13 különböző értékű kártyalap található. A játékstílusok igen különbözőek tudnak lenni. Megkölönböztetünk ante, illetve blind alaptétet. Ante alaptét esetén az asztalnál ülő összes játékos előre meghatározott tétet rak be. Blind alaptét esetén megkülönböztetünk kis és nagy vakot, amelyet körönként más és más játékosok raknak be. Minden esetben az osztótól közvetlenül balra ülő játékosnak kell beraknia a kis vakot. Az osztótól kettővel balra ülő játékosnak pedig a nagy vakot. Amíg az alaptétek nem kerültek be az asztalra, addig a ház nem kezdi meg a lapok kiosztását. \\
A \ref{fig:omaha}. ábrán látható az OMAHA játékstílus. A Texas Hold'Em játékstílushoz igen hasonló póker játék változat. A játékosok a kezükbe négy darab kártyalapot kapnak a vakok betétele után, majd megkezdődik az első licitkör, amelynek a végén három közös lap kerül ki az asztal közepére. Újabb licitkör kezdődik, amelynek a végén egy, majd az újabb licitkör végén, még egy újabb kártyalap kerül az asztal közepére, mint közöslap. A felhasználóknak maximum öt lapot lehet felhasználiuk a legjobb kéz előállítására. Az öt lapból kötelezően kettő a saját kézből történik a maradék három pedig a közös lapokból. Értelemszerűen a legerősebb kéz nyer \cite{card_combinations}. A játék célja, hogy minél több zsetont gyűjtsünk össze a partik során.
\begin{figure}[h!]
  \caption{OMAHA póker játék}
  \label{fig:omaha}
  \centering
    \includegraphics[width=\textwidth]{user-documentation/images/omaha.jpg}
\end{figure}

\subsection{Játékmenet}
Minden játékszerver úgy lett konfigurálva, hogy két játékos esetén a parti elkezdődjön. Ha valaki később csatlakozott az asztalhoz, az a megkezdett partiból semmit nem érzékel, mintha üres asztalnál ülne. Csak a következő partiba tud beszállni. Mindkét játékstílus esetén van egy BLIND kör, amikor a szerver bekéri a vakokat a játékosoktól. Az osztótól eggyel balra ülő játékos köteles betenni a kis vakot, a kis vaktól eggyel balra ülő játékos pedig köteles betenni a nagy vakot. Ebból a felhasználó nem lát semmit, köteles beadni a kis- vagy nagyvakot, amelyet egy automatizált eljárás hajt végre. Az osztó gomb  a legelső körben kerül kiosztásra a legelsőként csatlakozott játékoshoz. Az osztó gomb az óramutató járásával megegyező irányban halad. Minden új megkezdett parti esetén az osztó gomb a következő játékoshoz kerül. Minden parti legelső körében a kezdő játékos az osztótól balra ülő harmadik játékos, minden további kört a kis vakra kötelezett játékos kezd meg.  \\
A játékszerverek nem képesek kezelni, ha egy játékosnak elfogyott, illetve nincs elegendő zsetonja. Ebből kifolyólag esetenként előfordulhat negatív egyenleg, illetve nem definiált viselkedés. A játékosok szigorúan csak egymást követve küldhetnek utasításokat a szervernek, mindig az éppen soron levő játékos. A felhasználói grafikus felületen egyértelmű jelzéssel van ellátva az éppen soron levő játékos. A játékosok megadhatják, emelhetik a tétet, illetve, ha nem szeretnének az adott körben semmit csinálni, akkor CHECK-elhetnek. Az emelés mértéke a mindenkori játékasztal alaptét felét jelenti. Ugyanakkor lap eldobásra és a játék elhagyására is lehetőség van. A felhasználók korlátozottan visszanézhetik a korábbi leosztásokat, és a partiban történt eseményeket. A játékosok asztalonkénti maximum száma 5 fő. A kliensek a játék elhagyását követően újracsatlakozhatnak az adott játékszerverre a fentieket figyelembe véve. Lehetőség van asztalt váltani, és a játékszerverek (korlátozottan) képesek kezelni, ha a játékossal megszakad a kapcsolat. A kliens alkalmazások is korlátozott mértékben fel vannak készítve az esetleg kommunikációs hibákra.

\subsection{Játékstílusok} \label{subsubsec:game_styles}
A program kettő beépített játékstílust definiál
\begin{itemize}[leftmargin=2cm]
\item Classic \cite{five_card_draw}
\item Texas Hold'Em \cite{texas_holdem}
\end{itemize}

\pokerparagraph{Classic} 
A klasszikus játékstílus legfőbb ismertető jele, hogy mindenki öt lapot kap kézbe és nincsenek közös lapok. Először is a vakokra kötelezett játékosok rakják be a vakokat (automatizáltan), azután pedig úgynevezett előkör van, amikor a szerver minden játékosnak öt-öt lapot oszt kézbe, és ezek alapján lehet licitálni. Ha vége a körnek, akkor mindenki kicserélheti a lapjait, amiket saját maga választ ki a grafikus felületen a saját kártyalapjainak rákattintásával. Ha a kártyalap feljebb csúszott a grafikus megjelenítésen, akkor a kártyalapot cserére jelölte a játékos. A CHANGE feliratú gombbal cserélhetőek a kártyák. Mindenki akkor kapja meg az új kártyalapjait, ha már mindenki nyilatkozott a cseréről. Ezek után új kör indul, amikor is az új lapok birtokában tehetik meg a tétjeiket a játékosok. A kör után a szerver kihirdeti a nyertest, és a nyertes lapokat az asztal közepén jeleníti meg a grafikus felület. Kivétel, ha a nyertesek az éppen adott felhasználó, akkor a nyertes kártyalapok maga előtt jelen vannak, más kártyalapok nem kerülnek felfordításra. A felhasználók a nyertes lapok megtekintését követően kötelesek rákattintani a CHECK feliratú gombra, illetőleg ha kifutnak az időből, akkor a játék asztal kilépteti őket. Ha ebben a körben is mindenki nyilatkozott, akkor a játékasztal új partit indít.

\pokerparagraph{Texas Hold'Em}
A klasszikus játékstílussal erősen megegyező játékmenetű stílus. A különbség csupán annyi, hogy a ház 2-2 lapot oszt minden játékosnak, amelyekkel a játékosok a parti végéig rendelkeznek. Illetőleg a nyertes lapok semmilyen esetben sem középen, hanem a játékosoknál jelennek meg.

\section{Futtatás}

\subsection{A póker szerver elindítása}
A DVD lemezen a \path{\poker\release\} mappában található meg a poker-server-1.0.0.jar file. Nyissunk egy terminált a kijelölt könyvtárban, és adjuk ki a
 \begin{verbatim}
java -jar poker-server-1.0.0.jar
\end{verbatim}
parancsot. 
\begin{figure}[h!]
  \caption{Szerver}
  \label{fig:server_started}
  \centering
    \includegraphics{user-documentation/images/server_started.jpg}
\end{figure}
Ha a \ref{fig:server_started}. ábrának megfelelő konzol loglistát látunk, akkor a szervert sikeresen elindítottuk.

\subsection{A póker kliens elindítása}
A kliens futtatása hasonló módon történik, mint a szerveré. Navigáljunk a \path{\release\poker\kliens} mappába, és a konzolon adjuk ki a megfeleő parancsot.
Jöhet az ábra, meg a kódot kicsit átírni, hogy logoljon konzolra, mint a szerver...

\section{A póker játék használata} %Csatlakozás a játékhoz
\begin{figure}[h!]
  \caption{Bejelentkezési form}
  \label{fig:client_login}
  \centering
    \includegraphics[width=\textwidth]{user-documentation/images/login.jpg}
\end{figure}
A játék indítása után az \ref{fig:client_login}. ábrán látható bejelentkező felületet kapjuk. Ha még nem regisztráltuk magunkat a játékba, akkor kattinstunk a Regisztráció nevű gombra, amely felület a X. ábrán szerepel. Adjuk meg a regisztrálni kívánt felhasználó nevünket és jelszavunkat, majd regisztráljunk.  (Képek kellenek ide, error, info?) A szerver értesít minket, hogy a művelet sikeres, vagy sikertelen volt. Ügyeljünk arra, hogy névütközéseket a szerver nem enged. Tehát, ha valaki XYZ névvel már regisztrálva van, akkor még egy ugyanolyan nevű felhasználót nem enged regisztrálni a szerver. Továbbá győződjünk meg arról, hogy a megadott jelszavak megegyeznek.  \\
Sikeres regisztrációt követően a program visszairányít minket a bejelentkezési formhoz, amelyet értelemszerűen kitöltve be tudunk jelentkezni a programba.
A sikeres bejelentkezést követően a
\begin{figure}[h!]
  \caption{Játékasztalok}
  \label{fig:poker_tables}
  \centering
    \includegraphics[width=\textwidth]{user-documentation/images/tables.jpg}
\end{figure}
\ref{fig:poker_tables}. ábrán látható felület fogad minket, ahol például asztalhoz csatlakozhatunk. Tetszőlegesen válasszunk ki egy asztalt, majd kattintsunk a Csatlakozás feliratú gombra.
\begin{figure}[h!]
  \caption{Üres játékasztal}
  \label{fig:empty_table}
  \centering
    \includegraphics[width=\textwidth]{user-documentation/images/empty_table.jpg}
\end{figure}
A program átirányított minket a \ref{fig:empty_table}. ábrán jelölt felületre. Az üres játék asztal (optimális esetben) két dolgot jelenthet, vagy azt, hogy az asztalnál éppen játszanak, vagy pedig azt, hogy az asztalnál mi vagyunk az egyedüli játékosok. Indítsunk el egy második klienst (hivatkozás?) is, majd csatlakozzunk ugyanahhoz az asztalhoz, amelyiknél az előző kliens helyet foglalt. A játék (optimális esetben) elindult, és ha az asztal klasszikus játékstílusú, akkor
\begin{figure}[h!]
  \caption{Klasszikus játékstílusú asztal}
  \label{fig:classic_table}
  \centering
    \includegraphics[width=\textwidth]{user-documentation/images/classic_game.jpg}
\end{figure}
\ref{fig:classic_table}. ábrán látható felület fogad minket.

\clearpage
