%diploma-vaz fo dokumentum
%Keszult: 2004. februar
%(c) Markus Janos, http://mit.bme.hu/~markus, markus@mit.bme.hu
%Tesztelve: Miktex 2.3 alatt

%\documentclass[12pt,a4paper,twoside,openright]{report}  %Ketoldalas szedes
\documentclass[12pt,a4paper,oneside]{report}            %Egyoldalas szedes

%%Itt kivalaszthatjuk az egyes fejezeteket, ha nem akarjuk az egeszet forditani
%\includeonly{01_elozm, 02_eloszo, 03_1fej}

\usepackage{bmedipl}         %margo-, nyelv es egyeb allitas

%\usepackage{amsmath}         %matematikai segelycsomag
%\usepackage{graphicx}        %grafikai

\usepackage{setspace}         %1.5-os, 2-es sorkoz hasznalatahoz.
                              %Ettol a tablazatok, abrak, labjegyzetek maradnak 1-es sorkozzel!
\onehalfspacing               %1.5-os sortav. Nem kotelezo szerintem...
%\doublespacing                %2-es sortav. Csak korrekturahoz!...

%\usepackage[dcu]{harvard}    %harvard tipusu hivatkozashoz (ld. a dokumentaciojat)
% \citationstyle{dcu}
% \citationmode{abbr}
% \harvardparenthesis{square}
% \harvardyearparenthesis{round}
% \renewcommand{\harvardand}{\'es}


%a jelolt neve
\renewcommand{\nev}{Lapos Elem�r}

%konzulens adatai
\renewcommand{\konzulens}{dr.~F�lel� Ben�}
\renewcommand{\konzbeoszt}{egy.~docens}

%a dolgozat cime, ev
\renewcommand{\cim}{St�lusf�jlok k�sz�t�se v�gz�s hallgat�knak}
\renewcommand{\ev}{2004.}

\hyphenation{meg-szents�g-te-le-n�t-he-tet-len}  %egyedi elvalasztas

\begin{document}

%%%%%%%%%%%%%%%%%%%%%%%%%%%
% Szennycimoldal
%%%%%%%%%%%%%%%%%%%%%%%%%%%
 \begin{center}
 \pagenumbering{Roman}

 \vspace*{12mm}
 \LARGE{Diplomaterv}

 \vspace{8mm}
 \large{\nev}
 \vfill
 \ev
 \end{center}
 \thispagestyle{empty}
 \blankpage

%%%%%%%%%%%%%%%%%%%%%%%%%%%
% Diplomaterv-kiiras (ezt adjak, bele kell k�tni a diplom�ba)
%%%%%%%%%%%%%%%%%%%%%%%%%%%
\begin{center}

 \uppercase{Budapesti M�szaki �s Gazdas�gtudom�nyi Egyetem}\\
 \uppercase{Villamosm�rn�ki �s Informatikai Kar}\\
 \uppercase{M�r�stechnika �s Inform�ci�s Rendszerek Tansz�k}

 \vspace{6mm}
 \large\textbf{DIPLOMATERV FELADAT (ezt adj�k\dots)}

 \vspace{6mm}
 \textbf{\nev} \\ \strut \\
 szigorl� villamosm�rn�k hallgat� r�sz�re\\
 (nappali tagozat villamosm�rn�ki szak) \\

 \vspace{6mm}
 \textbf{\cim}\\
 (a feladat sz�vege a mell�kletben)

\end{center}

 \vspace{6mm}
 A tervfeladatot �ssze�ll�totta �s a tervfeladat
tansz�ki konzulense:

\begin{center}
\konzulens\\ \konzbeoszt
\end{center}

 \vspace{6mm}
 \begin{tabular}{p{80mm}l}
A z�r�vizsga t�rgyai:   & Els� t�rgy \\
                        & M�sodik t�rgy \\
                        & Harmadik t�rgy
 \end{tabular}

 \vspace{6mm}
 \begin{tabular}{p{80mm}l}
 A tervfeladat kiad�s�nak napja:         &  \\
 A tervfeladat bead�s�nak hat�rideje:    &
 \end{tabular}

\vfill

\begin{center}
\begin{tabular}{cc}
 \makebox[7cm]{\emph{dr.\ G�rg�nyi Andr�s}}    & \makebox[7cm]{\emph{dr.\ P�celi G�bor}} \\
 \makebox[7cm]{adjunktus, diplomaterv felel�s} & \makebox[7cm]{egyetemi tan�r, tansz�kvezet�}
\end{tabular}
\end{center}

 \vspace{6mm}
 \begin{tabular}{p{80mm}l}
 A tervet bevette:           & \\
 A terv bead�s�nak d�tuma:   & \\
 A terv b�r�l�ja:            &
 \end{tabular}


 \thispagestyle{empty}
 \blankpage


%%%%%%%%%%%%%%%%%%%%%%%%%%%
% Diplomaterv-kiiras melleklete (ezt is adjak, bele kell k�tni a diplom�ba)
%%%%%%%%%%%%%%%%%%%%%%%%%%%
\def\abstractname{Mell�klet}
\begin{abstract}

\begin{center}
\textbf{\cim}
\end{center}

Itt k�vetkezik a r�szletes feladatki�r�s

Szint�n el�re k�szen van, itt csak a helyet hagyjuk ki.


\begin{flushright}
 \vspace*{1cm}
 \makebox[7cm]{\konzulens}\\
 \makebox[7cm]{\konzbeoszt}
\end{flushright}
\end{abstract}

 \thispagestyle{empty}
 \blankpage

%%%%%%%%%%%%%%%%%%%%%%%%%%%
% Nyilatkozat
%%%%%%%%%%%%%%%%%%%%%%%%%%%
\def\abstractname{Nyilatkozat}
\begin{abstract}

Alul�rott  \emph{\nev},  a  Budapesti M�szaki �s Gazdas�gtudom�nyi
Egyetem hallgat�ja kijelentem,  hogy  ezt  a diplomatervet meg nem
engedett seg�ts�g n�lk�l, saj�t  magam k�sz�tettem,  �s a
diplomatervben csak a megadott forr�sokat haszn�ltam  fel. Minden
olyan  r�szt, amelyet  sz�  szerint,  vagy azonos �rtelemben, de
��tfogalmazva  m�s forr�sb�l ��tvettem, egy�rtelm�en, a forr�s
megad�s�val megjel�ltem.

\begin{flushright}
 \vspace*{1cm}
 \makebox[7cm]{\rule{6cm}{.4pt}}\\
 \makebox[7cm]{\emph{\nev}}\\
 \makebox[7cm]{hallgat�}
\end{flushright}
\end{abstract}

%%%%%%%%%%%%%%%%%%%%%%%%%%%
% Tartalomjegyzek
%%%%%%%%%%%%%%%%%%%%%%%%%%%
\tableofcontents

%%%%%%%%%%%%%%%%%%%%%%%%%%%
% Kivonat
%%%%%%%%%%%%%%%%%%%%%%%%%%%
\def\abstractname{Kivonat}
\begin{abstract}
\addcontentsline{toc}{chapter}{Kivonat}

A diplomaterv st�lusf�jlokkal foglalkozik.

Bemutatja, elemzi, kieg�sz�ti �s �sszefoglalja a st�lusok
haszn�lat�nak m�dj�t.

Egy- ill.\ k�toldalas opci�val is j�l m�k�dik.
\end{abstract}


%%%%%%%%%%%%%%%%%%%%%%%%%%%
% Abstract
%%%%%%%%%%%%%%%%%%%%%%%%%%%
\selectlanguage{english}
\begin{abstract}
\addcontentsline{toc}{chapter}{Abstract}

This master thesis discusses \LaTeXe\ style files for BUTE
undergraduate students.
\end{abstract}

\selectlanguage{magyar}
      %elso lapok (Cimlap, kiiras, tartalomjegyzek, Kivonat, Abstract,egyeb)
%Eloszo

\chapter*{El�sz�}

 \addcontentsline{toc}{chapter}{El�sz�}
 \markboth{\uppercase{El�sz�}}{\uppercase{El�sz�}}
 \pagenumbering{arabic}

Ebben a r�szben a diplomaterv ki�r�s elemz�se ker�lhet, t�rt�nelmi
el�zm�nyek, a feladat ki�r�s�nak indokl�sa, az eddigi megold�sok
(nagyon r�viden).

Ezen fel�l a diplomaterv fel�p�t�se (melyik fejezet mivel
foglalkozik).

Ide tehet� esetleges k�sz�netny�lv�n�t�s is.

\vfill
\newpage

Az el�sz� k�vetkez� oldala...

\vfill
\newpage

Ill. k�v. oldal.
     %Eloszo, ebben celkituzes, elozmenyek, felepites, koszonetnyilvanitas
                        %max. nehany oldal

\chapter{A zenei hangok pszichoakusztikai jellemz�i}

\section{Az ��lland�sult spektrum}

\section{Tranziens folyamatok}

\vfill
\newpage

K�vetkez� oldal teteje

\section{Sztochasztikus jelens�gek}

\section{K�ls� k�r�lm�nyek}

\vfill
\newpage

\section{Egyszer�s�t� t�nyez�k}

Ill.\ k�vetkez� oldal.
       %elso fejezet: bevezetes

\chapter{Az orgona hangja}


\section{Az orgona fel�p�t�se}

\strut

\section{A s�pok fizikai jellemz�i}

\strut

\subsection{Az ajaks�pok}

\strut

\subsection{A nyelvs�pok}

\strut

\section{A s�pok hangj�nak anal�zise}

\strut

\vfill
\newpage

K�v. oldal teteje.

\subsection{M�r�si k�r�lm�nyek}

\strut

\subsection{Az ��lland�sult spektrum}

\strut

\subsection{Sztochasztikus jelens�gek}

\strut

\subsection{Tranziens folyamatok}

\strut

\subsection{K�ls� k�r�lm�nyek}

\strut
       %masodik fejezet: a problema reszletes kifejtese

\chapter{A l�tez� modellek}


\section{Hammond-orgona}

\strut

\subsection{A szint�zis alapelve}

\strut

\subsection{Funkcion�lis egys�gek}

\strut

\subsection{A modell hangh�s�ge}

\strut

\section{Anal�g �ramk�r�s orgon�k}

\strut

\subsection{A szint�zis alapelve}

\strut

\subsection{Funkcion�lis egys�gek}

\strut

\subsection{A modell hangh�s�ge}

\strut

\section{A mintav�telez�ses elj�r�s}

\strut

\subsection{A szint�zis alapelve}

\strut

\subsection{Redundanciacs�kkent�s}

\strut

\subsection{A modell hangh�s�ge}

\strut

\section{A fizikai modellez�s}

\strut

\subsection{A szint�zis alapelve}

\strut

\subsection{Modellalkot�s}

\strut

\subsection{A modell hangh�s�ge}

\strut

\section{K�vetkeztet�sek}

\strut
       %harmadik fejezet: eddig alkalmazott megoldas


\chapter{A jelmodell alap� szint�zis}


\section{A koncepcion�lis jelmodell}

\strut

\subsection{A periodikus jel modellje}

\strut

\subsection{A jelmodell alkalmaz�sa hangszermodellez�sre}

\strut

\subsection{Az integr�lt jelmodell}

\strut




\section{A param�terek sz�rmaztat�sa}

\strut

\subsection{A spektrum meghat�roz�sa}

\strut

\subsection{Tranziens-modellez�s}

\strut

\subsection{A jellegzetes s�pzaj modellje}

\strut

\subsection{A vizsg�lt k�ls� hat�sok modellje}

\strut

\subsection{Hi�nyz� s�pok param�tereinek becsl�se}

\strut



\section{A jelmodell szimul�ci�ja}

\subsection{A szimul�ci� vez�rl�se}

\strut

\subsection{A diszkr�t komponensek el��ll�t�sa}

\strut

\subsection{Burkol�illeszt�s}

\strut

\subsection{A zaj implement�l�sa}

\strut

\subsection{A k�ls� k�r�lm�nyek figyelembev�tele}

\strut



\section{Val�s idej� implement�ci�}

\strut


\subsection{MIDI-parancs�rtelmez�s}

\strut

\subsection{Dinamikus er�forr�skioszt�s}

\strut

\subsection{Alapharmonikus el��ll�t�sa}

\strut

\subsection{Felharmonikusok gener�l�sa}

\strut

\subsection{Burkol�illeszt�s}

\strut

\subsection{A k�ls� k�r�lm�nyek figyelembev�tele}

\strut


\section{Az implement�ci�k min�s�t�se}

\strut
       %negyedik fejezet: a jelolt megoldasa
\chapter{�sszefoglal�s}

\section{Eredm�nyek}

\strut

\section{Tov�bbfejleszt�si lehet�s�gek}

\strut
       %Osszefoglalas: ertekeles, tovabbi munka

%%esetleg b�rmi ide. Pl. k�sz�netnyilv�n�t�s
     %Utolso lapok: koszonetnyilvanitas, egyeb...


%Megjegyz�s: c�lszer� haszn�lni BibTeX-et:

%(pl. egyszeru stilus:):
%\bibliography{mybib}
%\bibliographystyle{alpha}

%(pl. harvard st�lus -- ez esetben a harvard.sty is betoltendo):
%\bibliography{mybib}
%\bibliographystyle{dcu}


\begin{thebibliography}{Piszczalski11}
\addcontentsline{toc}{chapter}{\bibname}

\newcommand{\bi}{\bibitem}

\section*{K�nyvek}

\bi[Ellenhorst82]{ellenhorst}%
Ellenhorst, W., \emph{,,Handbuch der Orgelkunde I--II.,''} Frits
Knuf, Buren, 1975.

\section*{Disszert�ci�k, diplomatervek}

\bi[Angster90]{angsterkand}%
Angster J., \emph{,,Orgona ajaks�pok megsz�lal�s�nak �s rezg�s�nek
korszer� m�r�sei �s eredm�nyei''}, kandid�tusi �rtekez�s, MTA MMSz
Akusztikai Kutat�laborat�riuma, Budapest, 1990.

\section*{Cikkek, konferenciaanyagok}

\bi[P�celi86]{peceli}%
P�celi, G., ,,A common structure for recursive discrete
transforms,'' \emph{IEEE Transactions on Circuits and Systems},
CAS-33., 1035--36.~o., 1986.

\section*{El�ad�s-sorozatok}

\bi[Horv�thn�98]{horvathne}%
Horv�th Istv�nn�, \emph{,,M�szaki akusztika''} el�ad�ssorozat, BME
H�rad�stechnikai Tansz�k, BME VIHI 4107, Budapest, 1998.

\section*{Internet}

\bi[Rodgers]{rodgers}%
Rodgers organs, ,,Paralell digital imaging,''\\ URL:
\emph{http://www.rodgerscorp.com}\\
\emph{/features/pdi.html}, 1999.

\section*{Egy�b forr�sok}

\bi[Gravis]{gravis}%
Advanced Gravis Computer Technology Ltd.,
\emph{,,Gravis Ultrasound Plag \& Play User's Guide,''}
Appendix D (Technical Specifications), Gravis Ultrasound P\&P CD-ROM, 1996.

\end{thebibliography}

\emph{Megjegyz�s: c�lszer� haszn�lni BibTeX-et (l�sd a
forr�sf�jlban).}
       %Irodalomjegyzek

%Fuggelek

\appendix

\chapter*{F�ggel�k}
 \addcontentsline{toc}{chapter}{F�ggel�k}
 \markboth{\uppercase{F�ggel�k}}{\uppercase{F�ggel�k}}
%\chaptermark{F�ggel�k}

\setcounter{chapter}{6}     %F betu lesz

\blankpage
      %Fuggelek cimlap
\section{A m�rt orgon�k le�r�sa}

\subsection{Cs�sz�r}

\strut

\subsection{Nasz�ly}

\strut

\subsection{Tata}

\strut
      %1. fuggelek
\section{A m�r�eszk�z�k adatai}

\subsection{Mikrofonok}

\strut

\subsubsection{Akai ACM-50}

\strut

\subsubsection{AKG C-747}

\strut

\subsection{Technics RS BX-404 sztere� kazett�s deck}

\strut

\subsection{Gravis Ultrasound P\&P hangk�rtya}

\strut
      %2. fuggelek
\section{A kifejlesztett programok haszn�lata}

\subsection{Az anal�zis program}

\strut

\subsection{A Matlab szint�zis-program}

\strut

\subsection{A val�s-idej� DSP-program �s kell�kei}

\strut
      %3. fuggelek

\section{A CD-mell�klet tartalma}

\subsection{Sz�m�t�g�pes adatok}

\strut

\subsection{Audio demonstr�ci�k}

\strut
      %4. fuggelek
\section{Az ADSP-2181 EZ-KIT LITE k�rtya}

\strut
      %5. fuggelek
\section{A t�m�hoz kapcsol�d� �rdekesebb honlapok}

\strut
      %6. fuggelek
%\section{A diplomaterv b�r�lata}

\begin{center}
\large\textbf{\strut \\ B�r�lat}
\strut

\large{\nev: ,,\cim''}\\
\large{c�m� diplomaterv�r�l}

\end{center}

\noindent Szeretn�m el�rebocs�jtani,\dots

A v�lasztott\dots

\vspace*{1cm}

\noindent Budapest, 2004. j�nius 14.

\begin{flushright}
%\makebox[7cm]{\rule{6cm}{.4pt}}\\
\makebox[7cm]{\emph{B�r�l� B�la}}\\
\makebox[7cm]{okl.\ villamosm�rn�k}
\end{flushright}
      %Biralat: k�s�bbi k�t�shez, opcionalis

%%abrak, tablazatok jegyzeke

\addcontentsline{toc}{chapter}{�br�k jegyz�ke}
\listoffigures

\addcontentsline{toc}{chapter}{T�bl�zatok jegyz�ke}
\listoftables
        %Esetleges abrak, tablazatok jegyzeke (lehet az elejen is, nem kotelezo)

\chapter*{R�vid�t�sek}
 \addcontentsline{toc}{chapter}{R�vid�t�sek}
 \markboth{\uppercase{R�vid�t�sek}}{}

\selectlanguage{english}

\begin{tabular}{p{20mm}p{120mm}}

  ADC & Analog to Digital Converter \\
  AM  & Amplitude Modulation \\
  BME & Budapesti M�szaki Egyetem \\
  CAS & Circuits And Systems \\
  CCRMA  & Center for Computer Research in Music and  Acoustics \\
  CD  & Compact Disk \\
  DAC & Digital to Analog Converter \\
  DFT & Discrete Fourier Transformation \\
  DMA & Direct Memory Access \\
  FFT & Fast Fourier Transformation \\
  FIR & Finite Impulse Response (Filter) \\
  FM  & Frequency Modulation \\
  IC  & Integrated Circuit  \\
  ICASSP  & International Conference on Acoustics, Speech and Signal Processing\\
  IIR & Infinite  Impulse Response (Filter)\\
  IRCAM   & Institut de Recherche et Coordination Acoustique / Musique  \\
  JAES    & Journal of the Audio Engineering Society of America \\
  MIDI    & Musical Instruments Digital Interface \\
  MIPS    & Million Instructions Per Second \\
  MMSz    & M�szer- �s M�r�s�gyi Szakoszt�ly\\
  MPEG    & Moving Pictures Expert Group \\
  MTA & Magyar Tudom�nyos Akad�mia \\
  PCM & Pulse Code Modulation\\
  PM  & Physical Modeling  \\
  SNR & Signal to Noise Ratio \\
  THD & Total Harmonic Distortion

\end{tabular}

%\end{sloppypar}

\selectlanguage{magyar}
\newpage
\thispagestyle{empty}
\vspace*{\fill}
\begin{center}
\emph{Utols� oldal alja: ide j�het h�laad�s, logo, ISBN, stb.}
\end{center}
      %Roviditesek jegyzeke

\end{document}
