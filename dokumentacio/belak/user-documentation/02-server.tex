\subsection{Telepítés}
\subsubsection{Java SE Runtime Environment 8}
A programcsomag futtatásához legalább Windows XP operációs rendszer szükséges, amelyen Java SE Runtime Environment 8 futtató környezet \cite{jresite} (a továbbiakban: JRE) fut. A JRE feltelepítését követően manuálisan ellenőrizzük, hogy a rendszer felvette-e környezeti változóként az installációs könyvtárat. Navigáljunk az operációs rendszerben a környezeti változók módosítása panelhez, majd ellenőrizzük le, hogy a PATH nevű környezeti változóhoz hozzá lett-e adva az installációs könyvtár: \path{C:\Program Files\Java\jdk1.8.0_60\bin}. 
Ha nem, akkor pontosvesszővel (;) elválasztva egészítsük ki a változó értékét, majd indítsuk el a promptot (ha nyitva van, akkor indítsuk újra).
Ha a
\begin{verbatim}
java -version
\end{verbatim}
utasítás hatására az 
\begin{figure}[h!]
  \caption{JRE verzió}
  \label{fig:jre_version}
  \centering
    \includegraphics{user-documentation/images/java_version.jpg}
\end{figure}
 \ref{fig:jre_version}. ábrán látható szöveg jelenik meg a konzolon, akkor sikeres volt a JRE telepítése és beállítása. További instrukciókért ld. melléklet.
 
 \subsubsection{MySQL Community Server 5.6}
 A programcsomag megköveteli a MySQL Community Server 5.6 adatbázis-kezelő rendszer \cite{mysqlsite} (a továbbiakban: MySQL Server) használatát is. Letöltés után csomagoljuk ki a zip állományt egy tetszőleges könyvtárba, majd a fentiekkel megegyező módon adjuk hozzá a PATH nevű környezeti változó értékéhez a MySQL Server bin könyvtár elérési útvonalát. Ha ezzel végeztünk, akkor nyissük meg a promptot (ha nyitva van, akkor indítsuk újra), majd navigáljunk a MySQL Server bin könyvtárába, ott pedig adjuk ki a
 \begin{verbatim}
mysqld --install
\end{verbatim}
parancsot. A parancs végrehajtása után navigáljunk a szolgáltatások panelhez, amelyet a legkönnyebben a promptban a
 \begin{verbatim}
services.msc
\end{verbatim}
\begin{figure}[h!]
  \caption{MySQL Service}
  \label{fig:mysql_service}
  \centering
    \includegraphics{user-documentation/images/mysql_service.jpg}
\end{figure}
kiadott utasítással lehet elérni. Majd járjunk el a \ref{fig:mysql_service}. ábrának megfelelően. Térjünk vissza a konzolra, ahol adjuk ki a 
 \begin{verbatim}
mysql -u root -p
\end{verbatim}
parancsot, amely jelszót fog kérni. A beviteli sort hagyjuk üresen, nyomjunk entert. Ha sikeresen beléptünk az adatbázis-kezelő rendszerbe, akkor adjuk ki a
 \begin{verbatim}
SELECT VERSION();
\end{verbatim}
utasítást, és ha a 
\begin{figure}[h!]
  \caption{MySQL Service}
  \label{fig:mysql_service}
  \centering
    \includegraphics{user-documentation/images/mysql_version.jpg}
\end{figure}
\ref{fig:mysql_service}. ábrának megfelelő képernyőképet kapunk, akkor sikeresen feltelepítettük az adatbázis-kezelő rendszert.

\subsubsection{Az adatbázis használatba vétele}
Ha sikeresen elindítottuk a MySQL Servert, akkor szükségünk lesz egy új adatbázis sémára (és demo adatokra), amelyet a \path{X:\poker\release\poker-db.sql} állományban találunk. Ezt a filet kell lefuttatni az adatbázison, a hatása idempotens. A promptban adjuk ki a
 \begin{verbatim}
mysql -u root -p < X:\poker\release\poker-db.sql
\end{verbatim}
utasítást, amely jelszót fog kérni. A beviteli sort ugyancsak hagyjuk üresen. Ha sikeresen lefutott a parancs, akkor az adatbázis séma ``felhúzása'' megtörtént.

\subsection{A póker játékról}
A póker, mint kártyajáték igen népszerű szerencsejáték. Akár élő tv adásokat is végig lehet követni, ahol hatalmas főnyereményeket osztanak ki a dobogós helyezetteknek. Viszonylag sok fajtája terjedt el szerte a világon, kezdve a klasszikus 5 lapos leosztásokkal egészen az OMAHA-án át a jól ismert Texas Hold'Em játékstílusig. A játékot 52 lapos francia kártyapaklival játszák, amelyben 4 szín és 13 különböző értékű kártyalap található. A játékcélja, hogy minél több zsetont gyűjtsünk össze a partik során.

\subsubsection{Játékmenet}
Minden játékszerver úgy lett konfigurálva, hogy két játékos esetén a parti elkezdődjön. Ha valaki később csatlakozott az asztalhoz, az a megkezdett partiból semmit nem érzékel, mintha üres asztalnál ülne. Csak a következő partiba tud beszállni. Mindkét játékstílus esetén van egy BLIND kör, amikor a szerver bekéri a vakokat a játékosoktól. Az osztótól eggyel balra ülő játékos köteles betenni a kis vakot, a kis vaktól eggyel balra ülő játékos pedig köteles betenni a nagy vakot. Ebból a felhasználó nem lát semmit, köteles beadni a kis- vagy nagyvakot, amelyet egy automatizált eljárás hajt végre. Az osztó gomb  a legelső körben kerül kiosztásra a legelsőként csatlakozott játékoshoz. Az osztó gomb az óramutató járásával megegyező irányban halad. Minden új megkezdett parti esetén az osztó gomb a következő játékoshoz kerül. Minden parti legelső körében a kezdő játékos az osztótól balra ülő harmadik játékos, minden további kört a kis vakra kötelezett játékos kezd meg.  \\
A játékszerverek nem képesek kezelni, ha egy játékosnak elfogyott, illetve nincs elegendő zsetonja. Ebből kifolyólag esetenként előfordulhat negatív egyenleg, illetve nem definiált viselkedés. A játékosok szigorúan csak egymást követve küldhetnek utasításokat a szervernek, mindig az éppen soron levő játékos. A felhasználói grafikus felületen egyértelmű jelzéssel van ellátva az éppen soron levő játékos. A játékosok megadhatják (CALL), emelhetik (RAISE) a tétet, illetve, ha nem szeretnének az adott körben semmit csinálni, akkor CHECK-elhetnek. Az emelés mértéke a mindenkori játékasztal alaptét felét jelenti. Ugyanakkor lap eldobásra (FOLD) és a játék elhagyására (QUIT) is lehetőség van. A felhasználók (korlátozottan) visszanézhetik a korábbi leosztásokat, és a partiban történt eseményeket (LOG). A játékosok asztalonkénti maximum száma 5 fő. A kliensek a játék elhagyását követően újracsatlakozhatnak az adott játékszerverre a fentieket figyelembe véve. Lehetőség van asztalt váltani, és a játékszerverek (korlátozottan) képesek kezelni, ha a játékossal megszakad a kapcsolat. A kliens alkalmazások is (korlátozott mértékben) fel vannak készítve az esetleg kommunikációs hibákra.
\subsubsection{Játékstílusok} \label{subsubsec:game_styles}
A program kettő beépített játékstílust definiál
\begin{itemize}[leftmargin=2cm]
\item Classic \cite{five_card_draw}
\item Texas Hold'Em \cite{texas_holdem}
\end{itemize}

\pokerparagraph{Classic} 
A klasszikus játékstílus legfőbb ismertető jele, hogy mindenki öt lapot kap kézbe és nincsenek közös lapok. Először is a vakokra kötelezett játékosok rakják be a vakokat (automatizáltan), azután pedig úgynevezett pre-round van, amikor a szerver minden játékosnak öt-öt lapot oszott kézbe, és ezek alapján lehet licitálni. Ha vége a körnek, akkor mindenki kicserélheti a lapjait, amiket saját maga választ ki a grafikus felületen a saját kártyalapjainak rákattintásával. Ha a kártyalap ``feljebb'' csúszott a grafikus megjelenítésen, akkor a kártyalapot cserére jelölte a játékos. A CHANGE feliratú gombbal cserélhetőek a kártyák. Mindenki akkor kapja meg az új kártyalapjait, ha már mindenki nyilatkozott a cseréről. Ezek után új kör indul, amikor is az új lapok birtokában tehetik meg a tétjeiket a játékosok. A kör után a szerver kihirdeti a nyertest, és a nyertes lapokat az asztal közepén jeleníti meg a grafikus felület. Kivétel, ha a nyertesek az éppen adott felhasználó, akkor a nyertes kártyalapok maga előtt jelen vannak, más kártyalapok nem kerülnek felfordításra. A felhasználók a nyertes lapok megtekintését követően kötelesek rákattintani a CHECK feliratú gombra (vagy majd megcsinálom, hogy automatizált quit legyen ez is.... csak most még bugos.... winnercardsnál úgy is van nullpointerexceptionöm....). Ha ebben a körben is mindenki nyilatkozott, akkor a játékasztal új partit indít.

\pokerparagraph{Texas Hold'Em}
A klasszikus játékstílussal erősen megegyező játékmenetű stílus. A különbség csupán annyi, hogy a ház 2-2 lapot oszt minden játékosnak, amelyekkel a játékosok a parti végéig rendelkeznek. Illetőleg a nyertes lapok semmilyen esetben sem középen, hanem a játékosoknál jelennek meg.

\subsection{Futtatás}

\subsubsection{A póker szerver elindítása}
A DVD lemezen a \path{\poker\release\} mappában található meg a poker-server-1.0.0.jar file. Nyissunk egy terminált a kijelölt könyvtárban, és adjuk ki a
 \begin{verbatim}
java -jar poker-server-1.0.0.jar
\end{verbatim}
parancsot. 
\begin{figure}[h!]
  \caption{Szerver}
  \label{fig:server_started}
  \centering
    \includegraphics{user-documentation/images/server_started.jpg}
\end{figure}
Ha a \ref{fig:server_started}. ábrának megfelelő konzol loglistát látunk, akkor a szervert sikeresen elindítottuk.
\subsubsection{A póker kliens elindítása}
A kliens futtatása hasonló módon történik, mint a szerveré. Navigáljunk a \path{\release\poker\kliens} mappába, és a konzolon adjuk ki a megfeleő parancsot.
Jöhet az ábra, meg a kódot kicsit átírni, hogy logoljon konzolra, mint a szerver...

\subsection{A póker játék használata} %Csatlakozás a játékhoz
A játék 
\clearpage